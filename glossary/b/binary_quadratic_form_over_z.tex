 A function
 $ f :  \mathbb{Z}^2 \to  \mathbb{Z} $  of the form  $ f(x,y)=ax^2+bxy+c $  for some
 $ a,b,c \in  \mathbb{Z} $ .  We define the  discriminant of  $ f $  to be
 $  \Delta (f)=b^2-4ac $ .  If  $ f $  is  positive-definite (i.e.  it
takes strictly positive values except for  $ f(0,0)=0 $ ) then
 $  \Delta (f)<0 $ , and if  $ a,c>0 $  and  $  \Delta (f)<0 $  then  $ f $  must be
positive-definite.  Similarly,  $ f $  is  negative-definite iff
 $ a,c<0 $  and  $  \Delta (f)>0 $ .

