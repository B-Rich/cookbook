A constant, denoted B, which is the sum of the reciprocals of 
all twin primes:
\[ \mathrm{B} = \left ( \frac{1}{3}+ \frac{1}{5} \right )
+  \left ( \frac{1}{5}+ \frac{1}{7} \right )
+  \left ( \frac{1}{11}+ \frac{1}{13} \right )
+  \left ( \frac{1}{17}+ \frac{1}{19} \right )
+  \left ( \frac{1}{29}+ \frac{1}{31} \right )
+  ... \]
\par
(Note that since 5 occurs in two pairs of twin primes, it appears twice. Some
mathematicians do not do this.)
\par
It is approximately equal to 1.9021605820. The fact that this sequence converges
rather than diverging demonstrates that twin primes are scarce compared with 
ordinary primes (whose reciprocal sum diverges), but it does not tell us
whether the total number of twin primes is finite or infinite.