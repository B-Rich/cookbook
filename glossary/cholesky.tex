Major Andre-Louis Cholesky (1875-1918)
The mathematician after whom the Cholesky factorisation is named. He was born in France,
and worked in the Geodesic section of the Geographic service to the French army's artillery
branch. At this time the system of triangulation used in France, and based on the meridian
line of Paris, was being revised; new methods were needed in order to facilitate what was
not yet a quick or convenient process.
\par
Cholesky invented computation procedures based on the method of
least squares, for the solution of certain data-fitting problems in geodesy, to be put into practice in his triangulation of the French and British parts of Crete, and in his work in Algeria and Tunisia.
\par
His mathematical work was
posthumously published on his behalf in 1924 by a fellow officer,
Benoit.