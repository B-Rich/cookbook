If we have a standard normal variable, and a variable with a chi-squared
distribution with k degrees of freedom, then the ratio of the normal
variable to the square root of the chi-squared variable is another 
random variable, which has a distribution proportional to the t-distribution with k 
degrees of freedom.
\par
This is also the same as the square root of an F-distribution with 1, k degrees of
freedom, if k>0.
\par
i.e. given n observations $x_{i}$  from a normal distribution with
mean $ \mu , $ we define
\[ \mathrm{t} = \frac{\bar{\mathrm{x}} - \mu }
{\mathrm{s} / \sqrt{\mathrm{n}}} , \]
where 
\[ \mathrm{s}^2 = \sum _{\mathrm{i}=1} ^{\mathrm{n}}
\frac{ ( \mathrm{x}_{\mathrm{i}} - \bar{\mathrm{x}} ) ^{2} }
{\mathrm{n}-1} . \]
Then t has a t-distribution with n-1 degrees of freedom.
