Given a conic C and a point P, the polar of P is either
\par
1)the line defined as follows:
\par
Draw a line L through P, which cuts C at A and B. Take the tangents to C at 
A and B, and find their intersection X. The locus of X as L varies is a straight
line; this is the polar.
\par
or 2) the line defined as follows:
\par
Find the two tangents T and U of C which pass through P. The points where these
tangents touch C are called D and E. The line M joining D and E is the polar of
P with respect to C. (In this case P is called the pole of M with respect to C.)
\par
The second of these two is more commonly used - it is also known as the chord
of contact of tangents.