Pythagoras of Samos (c 569 BC - c 475 BC) - an ancient Greek
  mathematician. None of his writings survive, and we know relatively
  little about the mathematics which he did.
  He is said to have been the leader of a religious/scientific group and
  to have worked on arithmetic and geometry. The theorem which we know
  as Pythagoras' theorem was known long before Pythagoras' time,
  although Pythagoras himself might have been the first to prove it.
  Pythagoras is said to have been one of the first to believe that the
  world is fundamentally mathematical, that phenomena are ruled by
  mathematical processes. He was interested in the abstract nature of
  mathematical concepts such as the natural numbers, or the idea of a
  proof.
  The Pythagoreans are said to have discovered the irrational numbers,
  the five regular (Platonic) solids, and to have shown that the sum of
  the angles in a triangle is equal to two right angles.