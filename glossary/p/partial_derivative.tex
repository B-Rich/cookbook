If f(x, y) is a function of both x and y, then the partial derivative of
f with respect to x, written $ \frac{\partial \mathrm{f}}{\partial \mathrm{x}}, $ is:
\[ \frac{\partial \mathrm{f}}{\partial \mathrm{x}}
= \lim_{\mathrm{h} \to 0} \frac {\mathrm{f(x + h, y) - f(x, y)}}{\mathrm{h}} \]
\par
This is equivalent to differentiating with respect to x and treating y as though
it were a constant. It can also be written $f_{x}.$ 
\par
For example, if $ \mathrm{f(x, y)} = \mathrm{x}^2 + \mathrm{xy} + 2 \mathrm{y}^2, $ 
\par
then $ \frac{\partial \mathrm{f}}{\partial \mathrm{x}} = 2 \mathrm{x + y} $
\par
We can define partial derivatives with respect to z, and of functions of 
three or more variables, in the same way.