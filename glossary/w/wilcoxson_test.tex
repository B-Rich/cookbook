To test whether two sets of data are likely to be from the same population,
we put the two sets of data together into rank order, and then replace each
piece of data with its rank. Then we find the sum of the ranks of the data
from one of the original sets.
\par
Tables of critical values for various sizes of samples are published.
\par
For example, if the two sets of data were: A: {1, 3, 5, 4, 6, 2, 7} and B: 
{3, 6, 7, 4, 8, 4} then we would rank them as follows: 1(A), 2(A), 3(A), 3(B),
4(A), 4(B), 4(B), 5(A), 6(A), 6(B), 7(A), 7(B), 8(B). Replacing each
number with its rank gives: 1(A), 2(A), 3(A), 4(B), 5(A), 6(B), 7(B), 8(A),
9(A), 10(B), 11(A), 12(B), 13(B).
\par
Then we find the sum of the ranks of the values from A: 1+2+3+5+8+9+11=39;
we would compare this with a published critical value.