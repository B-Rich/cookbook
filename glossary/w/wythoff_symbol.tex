A way of specifying any uniform polyhedron using three rational numbers and
a bar |. If two of the numbers are equal to 2, the third is arbitrary. Otherwise
we have the restriction that the numerators can only be 2, 3, 4, or 5 - and 4
and 5 are not allowed to occur together. There are a few cases:
\par
\textbf{Regular polyhedra} have a pattern {r, r, r, ..., r} of polygons round each
vertex (where {3, 3, 3, 3} would mean triangle-triangle-triangle-triangle).
By inserting 'non-faces' with two sides, we can write this as {2, r, 2, r, ..., 2, r}.
We write this as p|2r, where p is the number of times the pattern is repeated 
(for example, {3, 3, 3, 3} would become {2, 3, 2, 3, 2, 3, 2, 3} and be written
4|23).
\par
\textbf{Quasi-regular polyhedra} have a pattern {q, r, q, r, ..., q, r}
of polygons round each vertex, (where {3, 4, 3, 4} would mean 
triangle-square-triangle-square). We write this as p|qr , where p is the number
of times the pattern repeats itself (so {3, 4, 3, 4} would be written 2|34).
\par
\textbf{Semi-regular polyhedra} have pattern {p, 2r, q, 2r} of polygons round each
vertex, where p may be 2, to represent a 'non-face'. This is written with
Wythoff symbol pq|r.
\par
\textbf{Even-faced polyhedra} have pattern {2p, 2q, 2r} of polygons round each vertex.
This is written with Wythoff symbol pqr|.
\par
\textbf{Snub polyhedra} have pattern {3, p, 3, q, 3, r} of polygons round each vertex.
This is written with Wythoff symbol |pqr.
\par
\textbf{The Great Dirhombicosidodecahedron} has eight faces round each vertex;
the Wythoff method can only deal with up to 6 faces round a vertex, so there is no
true Wythoff symbol for this uniform polyhedron. It is denoted with the
pseudo-Wythoff symbol (|3/2 5/3 3 5/2).
\par
For the purpose of making Wythoff symbols, a fraction a/b means a star made
from a points by joining every $b^{th}$  point. For instance, a five-pointed
star can be made from five points by joining up every second point; this is written
5/2.