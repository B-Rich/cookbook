Tetrahedral numbers are the numbers which can be made by considering a tetrahedral
pattern of beads in three dimensions.
\par
For example: if we make a triangle of beads with 3 beads to a side,
and on top of this we place a triangle with two beads to a side,
and on top of that a triangle with one bead to a side, we have made 
a tetrahedron of beads. In this case the total number of beads is
(3rd triangular number)+(2nd triangular number)+(1st triangular number)
=6+3+1=10.
\par
In general the $n^{th}$  tetrahedral number is equal to the sum of the
first n triangular numbers. This is the same as the 4th number from the left
in the $(n+3)^{th}$  row of Pascal's triangle.
\par
We can use the binomial formula for numbers in Pascal's triangle to
show that the nth tetrahedral number is $$^{n+2}C_{3},$$  
or (n+2)(n+1)n/6.
\par
The only numbers which are both tetrahedral and square are
4 $$(=2^{2}=T_{2}) and 19600 (=140^{2}=T_{48}).$$  
