A method of representing numbers, widely used in computing, using
  two sequences of digits - the significant digits of the number, and
  the position of the radix point.  
  A number $N$ is represented as $N = r \times b^e$, where $r$ is a real
  number, usually constrained to be within a range (often $[1/b, 1]$),
  and $b$ is the base of the representation (an integer), and $e$ is
  another integer. $r$ is called the mantissa, and $e$ the exponent.  
  For example, 365.24 can be represented as $0.36524 \times 10^3$ 
  - scientific notation is special case of
  floating-point notation with base 10.  
  In computing the base 2 is often used, and $e$ must lie in a given
  range, and $r$ is constrained to have a particular number of
  digits. This means that when numbers are multiplied, etc, rounding
  errors can occur due to the extra digits of $r$ being discarded.