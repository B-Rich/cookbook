To factorise a polynomial is to write it as the product of a set of
smaller polynomials (ie polynomials whose highest power of x is smaller
than that of the polynomial we started with).
\par
For instance we might factorise $x^{2}-3x-10$  by writing it as
(x-5)(x+2).
\par
Or we can factorise $x^{3}-2x^{2}-15x by writing it as (x^{2}+3x)(x-5)$ 
or $(x^{2}-5x)(x+3)$  or x(x-5)(x+3), etc.
\par
If c is a root of the polynomial then (x-c) is a factor of it; for quadratic,
cubic and quartic polynomials this can be useful for factorising. But there is no
general way to factorise any polynomial quickly.