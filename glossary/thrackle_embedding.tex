For a given graph, a thrackle embedding is a function which 
turns each vertex of the graph into a point, and each edge
of the graph into a continuous line, with the conditions that:
\par
Lines do not cross themselves;
\par
Each pair of lines intersects exactly once;
\par
Two vertices connected by an edge will always be turned into
two endpoints and a line;
\par
Two lines which intersect must cross at the point of intersection -
or they must meet at a common endpoint.

\par
In the 1960s, Conway conjectured that there is no graph which has 
more edges than vertices and has a thrackle embedding.