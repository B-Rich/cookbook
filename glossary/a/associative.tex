An operation is associative if, when the operation appears twice in
an expression, it doesn't matter where you put the brackets.
\par
For instance, addition is associative because 
$$(a+b)+c \equiv a+(b+c)$$
This is called the associative law.
Multiplication is associative because
$$(a b)c \equiv a(b c)$$
Division and subtraction are not associative since
$$(a-b)-c \not \equiv a - (b-c)$$
and
$$\frac{\left(\frac{a}{b}\right)}{c} \not \equiv \frac{a}{\left(\frac{b}{c}\right)}$$