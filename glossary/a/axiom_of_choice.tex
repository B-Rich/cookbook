For any set $S$ there is a function $f$ with the property that if $X$
is a nonempty subset of $S$, then $f(X)$ is in $X$. This means that
$f$ (called the 'choice function') effectively chooses an element from
the set on which it acts.
This axiom is independent of and consistent with the other axioms of
set theory, but is often thought to be counterintuitive (in this
respect it is like the parallel postulate in geometry).