\[ \lim_{ \mathrm{x} \to \mathrm{a}} \frac {\mathrm{f(x)}}{\mathrm{g(x)}} = 
\frac{\displaystyle{\lim_{\mathrm{x} \to \mathrm{a}}}\frac{\mathrm{df}}{\mathrm{dx}}}
{\displaystyle{\lim_{\mathrm{x} \to \mathrm{a}}}\frac{\mathrm{dg}}{\mathrm{dx}}} \]
\par
This is a useful rule for finding the limit of a ratio, if both
parts of the ratio go to zero at the limit;
\par
eg, \[ \lim_{\mathrm{x} \to 1} \frac {\mathrm{x}^2 - 1}{\mathrm{x}^2 - 3 \mathrm{x} + 2} = 
\frac{\displaystyle{\lim_{\mathrm{x} \to 1}}2 \mathrm{x}}
{\displaystyle{\lim_{\mathrm{x} \to 1}}2 \mathrm{x} - 3} \]
\[ = \frac{2}{-1} \]
\[ = -2 \]
\par
This is known as l'H\^opital's rule.