Given sequences  $ (a_n) $  and  $ (b_n) $ , if there is a constant  $ M $  such
that  \[|a_n| \leq M|b_n| \quad \text{for all }n, \]
then we write  $ a_n = O(b_n) $ , read ` $ a_n $  is big Oh of  $ b_n $ '.
Similarly for two functions on a set. If  $ a_n=O(b_n) $  and
 $ b_n=O(a_n) $  we write  $ a_n =  \Theta (b_n) $ . If
 \[ \lim_{n \to  \infty } \left| \frac{a_n}{b_n} \right| = 0, \]
then  $ a_n $  is said to be  $ o(b_n) $  as  $ n \to \infty  $ . Compare
asymptotically equivalent.


