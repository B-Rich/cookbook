If we were going to integrate numerically to find the area under a curve, we would
divide the area into a lot of thin strips, and then use the equation of the curve to 
work out the co-ordinates of the corners of the strips. From this we can work out
the approximate area of each strip, and then add up all of these areas in order to
find the total area under the curve.
\par
  The same kind of process can be used to do other integrations, like finding the
length of a curve or the volume of a solid.
\par
Numerical integration is useful when it is hard or impossible to find, 
algebraically, the integral of the relevant function.