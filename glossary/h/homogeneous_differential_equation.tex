A first-order differential equation is called homogeneous if it has
the form $ \frac{\mathrm{dy}}{\mathrm{dx}} = \mathrm{f} \left ( \frac{\mathrm{y}}
{\mathrm{x}} \right ) . $
\par
A second-order differential equation is called homogeneous if it has the form
$ \frac{\mathrm{d}^2 \mathrm{y}}{\mathrm{dx} ^2} + \mathrm{p(x)}
\frac{\mathrm{dy}}{\mathrm{dx}} + \mathrm{q(t)y} = 0 . $
\par
In general, a homogeneous differential equation of order n has the form
$ \frac{\mathrm{d}^{\mathrm{n}} \mathrm{y}}{\mathrm{dx}^{\mathrm{n}}}
+ \mathrm{p}_1 ( \mathrm{x} ) 
\frac{\mathrm{d}^{\mathrm{n-1}} \mathrm{y}}{\mathrm{dx}^{\mathrm{n-1}}}
+ ...
+ \mathrm{p}_{\mathrm{n}-1} ( \mathrm{x} ) \frac{\mathrm{dy}}{\mathrm{dx}}
+ \mathrm{p}_{\mathrm{n}} ( \mathrm{x} ) \mathrm{y} = 0 . $