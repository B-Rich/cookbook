If Achilles and a tortoise have a race, and Achilles gives the
tortoise a headstart of fifty metres, then Achilles will never
catch up with the tortoise. Why? 
\par
Because first, Achilles must get to where the tortoise started
from, point A, say - but by the time he does this the tortoise
will have moved to another point, B. So, next, Achilles must get
to B. But by then, the tortoise has got to yet another point, C.
This argument can go on forever, and Achilles will never be in 
the same place as the tortoise.
\par
This is called a paradox, because reasoning which looks plausible
gives us a conclusion which is obviously wrong.
Some of the ancient Greek philosophers said that Zeno was wrong
because time cannot be infinitely subdivided; eventually the 
lengths of time for Achilles to get from one point to another
gets so small that they are meaningless, and the argument cannot
go on. But this doesn't quite solve the problem...
\par
Nowadays, we can use the mathematics of geometric series to show
that the infinite number of short lengths of time add up to a
finite total, so that Achilles catches up with the tortoise after
a definite length of time.