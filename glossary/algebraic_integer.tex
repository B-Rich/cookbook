A subset of a number field  $ K/ \mathbf{Q} $  that
plays a role similar to that played by  $  \mathbf{Z} $  in  $  \mathbf{Q} $ . An element  $ a $ 
of  $ K $  is an algebraic integer if there is a monic polynomial
 $ f(X) \in \mathbf{Z}[X] $  such that  $ f(a)=0 $ . The set of such numbers is written
 $  \mathcal{O}_{K} $ ; an equivalent characterisation of the property
 $ a \in  \mathcal{O}_{K} $  is that the NMP of  $ a $  lies in  $  \mathbf{Z}[X] $ . Also,
an algebraic integer has norm and trace in  $  \mathbf{Z} $ .
The ``usual integers''  $  \mathbf{Z} $  may be referred to as  rational
integers, as  $  \mathbf{Z}= \mathcal{O}_{ \mathbf{Q}} $ .
The set of algebraic integers  $  \mathcal{O}_{K} $  is a subring of  $ K $ ,
and also if  $  \beta  $  is a root of a monic polynomial with
coefficients in  $  \mathcal{O}_{K} $ , then  $  \beta  $  is also an algebraic
integer. We may always find an  integral basis for  $  \mathcal{O}_{K} $ ,
i.e.  a set  $  { \alpha _i } $  of algebraic integers such that
 \[ \mathcal{O}_{K}= \mathbf{Z} \alpha _1+ \ldots+ \mathbf{Z} \alpha _n, \] where  $ n=[K: \mathbf{Q}] $ .
For example, for the quadratic field  $ K= \mathbf{Q}( \sqrt{d}) $ ,
 $ d \in  \mathbf{Z} \setminus  {0,1 } $ square free, we have
 \[ \mathcal{O}_{K}= \left \{ \array { 
 \{u+v \sqrt{d} \mid u,v \in  \mathbf{Z} \}= \mathbf{Z}+ \mathbf{Z} \sqrt{d} ,  d \neq 1 (mod 4) \ 
 \{u+v \sqrt{d} \mid u,v \in  \mathbf{Z}\quad \text{ or }u,v \in  \{ \mathbf{Z}+ \frac{1}{2} \} \} =  \mathbf{Z}+ \mathbf{Z} \frac{1+ \sqrt{d}}{2} , d \equiv 1 (mod 4) }
\right
 \]