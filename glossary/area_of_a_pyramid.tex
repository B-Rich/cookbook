A pyramid whose base is a regular $n$-sided polygon, whose apex is directly above
the centre of the base, has total surface area $nA+B$, where $A$ is the area
of one of the slanting faces, and $B$ is the area of the base.
\par
The area of a regular n-gon is 
$$ \frac{1}{4} x^2 n \tan (90 - \frac{180}{n} ) , $$ 
where $x$ is the 
length of one of its edges - so this is the area of the base.
\par
And the face height of a pyramid is 
$$ f = \sqrt{h^2 + \frac{1}{4} x^2 \tan ^2 (90 - \frac{180}{n} ) } , $$ 
where 
$h$ is the height of the pyramid.
\par
So each face has area $xf/2$ (half base $\times$ height).
\par
Putting all of these pieces together, the total area of the pyramid
is:
\[ \frac{n}{2}x \sqrt{h^2 + \frac{1}{4} x^2 \tan ^2 \theta }
+ \frac{1}{4} x^2 n \tan \theta , \] 
where $ \theta = 90 - \frac{180}{n}. $