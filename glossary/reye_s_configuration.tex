 Three sets of four concurrent lines, meeting in threes to form
a hexahedron (think of a perspective drawing of a cube, 
with three 'vanishing points'), together with the four internal
diagonals of the hexahedron and their point of intersection.
This configuration of 16 lines and 12 points, with 4 lines through
each point and 3 points on each line, is called Reye's configuration.
\par
Alternatively, consider the 6 plane faces of this hexahedron, together
with the six diagonal planes passing through pairs of opposite 
edges. This configuration of 12 planes and 12 points (6 points
on each plane, 6 planes through each point) is called Reye's configuration.
