 1) A particle continues in a state of rest of uniform motion in a
straight line unless a force acts on it.
\par
2)If a resultant force acts on a particle, the particle's
linear momentum changes at a rate proportional to the size of the
force and in the same direction as the force: $ F = \frac{d}{dt}(mv) . $
(In the special case where the mass is constant, this becomes F=ma.
\par
3)When one object exerts a force upon another by contact, there is 
always a force of reaction equal in size and opposite in direction 
to the acting force.
