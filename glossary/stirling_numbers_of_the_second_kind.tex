If we have a set of n elements, and we want to partition it into
k subsets each with at least one element, the number of ways to do this
is called the Stirling number of the second kind, 
$ \mathrm{S}_{\mathrm{n}} ^{( \mathrm{k} ) } $ or 
$ \left \{ 
\begin{array}{c} 
  \mathrm{n} \\
  \mathrm{k}
\end{array} 
\right \} . $
\par
For example if we have the set {1,2,3,4} we can partition it into 
two subsets in the following ways:
\par
{{1},{2,3,4}}
\par
{{2},{1,3,4}}
\par
{{3},{1,2,4}}
\par
{{4},{1,2,3}}
\par
{{1,2},{3,4}}
\par
{{1,3},{2,4}}
\par
{{1,4},{2,3}}
\par
So $ \mathrm{S}_4 ^{(2)} = 7 . $