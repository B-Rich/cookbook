This is the method for solving any quartic equation.
\par
Say we have:
\par $ax^{4}+bx^{3}+cx^{2}+dx+e=0.$ 
\par
Divide by a, to rewrite this as:
$$x^{4}+fx^{3}+gx^{2}+hx+j=0.$$ 
Put $y=x+f/4, so x=y-f/4$, so 
\begin{eqnarray*}
  x^{4}&=&y^{4}-f^{3}y+3f^{2}y^{2}/8-f^{3}y/16+f^{4}/256 \\
  x^{3}&=&y^{3}-3y^{2}f/4+3yf^{2}/16-f^{3}/64 \\
  x^{2}&=&y^{2}-2yf/4+f^{2}/16
\end{eqnarray*}
so $x^{4}+fx^{3}+gx^{2}+hx+j=y^{4}-fy^{3}+3f^{2}/y^{2}/8$  $-f^{3}y/16+f^{4}/256+fy^{3}-3y^{2}f^{2}/4+3yf^{3}/16-f^{4}/64$  $+gy^{2}-gyf/2+gf^{2}/16+hy-hf/4+j$ 
\par $=y^{4}+y^{3}(-f+f)$  $y^{2}(3f^{2}/8)-3^{2}/4+4)$  $+y(-f^{3}/16+3f^{3}/16-gf/2+h)$  $+(f^{4}/256-f^{4}/64+gf^{2}/16-hf/4+j)$ 
\par
So we can rewrite the original equation as:
\par $y^{4}+py^{2}+qy+r=0$ 
\par
with $p=-3f^{2}/8+g,$  
\par $q=f^{3}/8-gf/2+h,$  
\par$r=-3f^{4}/256+gf^{2}/16-hf/4+j.$ 
Now, we can write:
\par $y^{4}+py^{2}=-qy-r$  
\par $y^{4}+2py^{2}+p^{2}=py^{2}-qy-r+p^{2}$
\par $(y^{2}+p)^{2}=py^{2}-qy-r+p^{2}.$ 
\par
Now, for any z,
\par $(y^{2}+p+z)^{2}=((y^{2}+p)+z)^{2}$  
\par $=(y^{2}+p)^{2}+2(y^{2}+p)z+z^{2}$  
\par $=py^{2}-qy-r+p^{2}+2z(y^{2}+p)+z^{2}$  
\par $=(p+2z)y^{2}-qy+(p^{2}-r+2pz+z^{2})$   (*)
\par
The right hand side of (*) is a quadratic in $y$; and we can choose $z$
so that it is a perfect square, i.e. so that the discriminant is zero, ie:
\par $(-q)^{2}-4(p+2z)(p^{2}-r+2pz+z^{2})=0.$ 
\par
We can rewrite this as $(q^{2}-4p^{3}+4pr)+(-16p^{2}+8r)z-20pz^{2}-8z^{3}=0.$ 
\par
This is a cubic equation in $z$. 
So we can solve it for $z$ using the cubic equation
formula (Cardano's method).
\par 
When we have solved this to find a value for $z$, we can
substitute in this value of $z$ to (*). This makes the right hand side
of (*) a perfect square, so we can take the square root of both sides
of (*). (*) is then a quadratic equation in $y$, which we can solve
using the quadratic solution formula. The values of $y$ will easily give
us values of $x$, i.e. solutions of the original equation.  
\par There can be 0, 1, 2, 3, or 4 solutions.