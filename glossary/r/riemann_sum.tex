We can find the area under a function f(x) between endpoints x=a and x=b in the 
following way:
\par
Divide the interval [a, b] into n parts by points $x_{0}, x_{1},$ 
... $x_{n}, with x_{0}=a and x_{n}=b.$  The subintervals have
lengths $x_{1} - x_{0} ... x_{n} - x_{n-1}.$  Pick a 
point $t_{i} in each interval [x_{i}, x_{i+1}].$ 
\par
Then a Riemann Sum is
\[ \mathrm{R} = \sum _{\mathrm{i} =0} ^{\mathrm{n} - 1 } 
( \mathrm{x}_{\mathrm{i} +1} - \mathrm{x} _{\mathrm{i}} ) \mathrm{f(t} _{\mathrm{i}} ) \]
Then we can define $ \int _{\mathrm{a}} ^{\mathrm{b}} \mathrm{f(x)dx} $ as 
the limit of R as the size of the largest subinterval goes to zero.