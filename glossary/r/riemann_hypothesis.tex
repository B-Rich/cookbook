The hypothesis that the Riemann zeta function 
$ \zeta ( \mathrm{s} ) = \sum ^{\infty} _{\mathrm{n}=1} \mathrm{n} ^{- \mathrm{s}} $
is equal to zero only when s (which is a complex number) is equal to
-2k for some natural number k, or when the real part of s is 1/2.
\par
This has not been proved, although it has been shown that there are infinitely
many values of s with real part equal to 1/2 for which the Riemann zeta function 
is equal to zero.