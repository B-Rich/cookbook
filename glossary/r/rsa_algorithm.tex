This is the algorithm which is very widely used for encrypting all kinds of
  private messages:
  \begin{itemize}
  \item Find two large prime numbers $P$ and $Q$;
  \item Choose a number $E$ less than $PQ$, which has no prime factors
  in common with $(P-1)(Q-1)$;
  \item Find $E$, the multiplicative inverse of $D mod
    (P-1)(Q-1)$. This means that $DE=1 (mod(P-1)(Q-1))$, i.e. $(DE-1)$
    is divisible by (P-1)(Q-1);
  \item Now the function to encrypt a message represented by a
    positive integer $T$, is $f(T)=T^{E}(mod(PQ)).$
  \item The function to decrypt an encrypted message represented by
  $C$, is $g(C)=C^{D}(mod(PQ)).$
  \end{itemize}
The \emph{public key} is the pair of numbers $(PQ, E)$. This can be published
freely.  Your \emph{private key} is the number $D$, and must be kept
secret. This means that anyone can encrypt messages to me using my
public key, but only I can read them using my private key.
This works because there is no known way to work out $D$, $P$ or $Q$ given
$(PQ, E)$, except to factorise $PQ$. If $P$ and $Q$ each have around 1024
digits, in binary, this factorisation would take billions of years
using present-day computers.