Twelve is the first number which can be written as a product of its proper divisors
in more than one way: $ 12=2\cdot 6= 3\cdot 4. $
\par
It is also the first abundant number.
\par
It is the number of edges of both the cube and the octahedron.
\par
It is the first abundant number and therefore the first
superabundant number.
\par
It is the number of faces of the dodecahedron, a Platonic solid,
and the number of vertices of the icosahedron.
\par
It is the third pentagonal number.
\par
$ \phi(12)=4, \quad d(12)=6, \quad \sigma(12)=28 . $