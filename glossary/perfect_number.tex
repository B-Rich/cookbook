A whole number which is equal to the sum of all its factors except itself. 
For example, 6 has the factors  1, 2, 3 and 6.  
Ignoring the six and adding the rest of the factors, 1 + 2 + 3, gives 6.
\par
Likewise:
28 = 1+2+4+7+14

\par
The next two perfect numbers are 496 and 8128.

\par
There are not many known perfect numbers.
\par
The first five are 6, 28, 496, 8128, 33550336. All perfect numbers end
in 6 or 8 in base 10, but the 6s and 8s do not alternate.
All perfect numbers are of the form $2^{n-1}(2^{n}-1)$ 
where $2^{n}-1$  is a Mersenne prime. 
\par
It is not known whether there are infinitely many perfect numbers,
nor is it known whether there is an odd perfect number.
