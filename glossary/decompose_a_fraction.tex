To decompose a fraction is to write it as the sum of a set of partial fractions.
The way to do this is first to factorise the denominator, for instance we decompose
$ \frac{1}{6} $ by noting that \ensuremath{6=2 \times 3,} so $ \frac{1}{6} = \frac{1}{2} \times
\frac{1}{3} . $ This means that we can write $ \frac{1}{6} = \mathrm{A} \frac{1}{2}
+ \mathrm{B} \frac{1}{3} . $
\par
We can find the values of A and B:
\[ \frac{1}{6} = \mathrm{A} \frac{1}{2} + \mathrm{B} \frac{1}{3} 
= \frac{ 3 \mathrm{A} + 2 \mathrm{B} } {6} \]
\[ \frac{1}{6} = \frac{ 3 \mathrm{A} + 2 \mathrm{B} } {6} \]
\[ 1 = 3 \mathrm{A} + 2 \mathrm{B} \] 
\par
- so one solution is A=1, B=-1.
\par
So, $ \frac{1}{6} = \frac{1}{2} - \frac{1}{3} . $
\par
This method can also work when the fraction has polynomials as numerator 
and denominator.