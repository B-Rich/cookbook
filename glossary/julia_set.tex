 The iterated mapping $ z \mapsto z^2 + k $ has a strange attractor
for certain values of k. This attractor is called a Julia set. 
\par
For other values of k the mapping goes to infinity or to a fixed point.
Values of k for which the mapping does not go to infinity are said
to lie in the Mandelbrot set.
\par
If the starting point of the mapping lies in this Mandelbrot set,
then the Julia set will be a connected set. Otherwise it will be
disconnected dust, called Fatou dust.
\par
Other mappings may be used, for instance $ z \mapsto a \cos z + k ; $
the strange attractors of these mappings are also called Julia sets,
though they do not have the same connection with the Mandelbrot set.
