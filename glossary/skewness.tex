A measure of how asymmetric a distribution is. There are several such 
measures, but they share the feature that a long tail on the right is
called positive skewness and a long tail on the left is called 
negative skewness. Skewness attempts to measure how much of the
'weight' of the distribution lies to the right of the mean.
\par
The coefficient of skewness is $ \gamma _1 = \mu _3 / \mu _2 ^{3/2} , $
where $ \mu _i $ is the $i^{th}$  moment about the mean.
\par
Other measures of skewness are:
\par
(mean-mode)/(standard deviation);
\par
\[ \frac{Q_3 - 2Q_2 + Q_1 }{Q_3 - Q_1} , \] where $Q_{1},$  $Q_{2} and Q_{3}$  are the first quartile, median and
third quartile respectively.