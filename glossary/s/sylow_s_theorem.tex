Let  $ p $  be a prime and group  $ G $  be of order  $ p^ms $  with
 $ (p,s)=1 $ . Then we call a subgroup  $ H $  of  $ G $  a  Sylow
 $ p $ -subgroup of  $ G $  if  $ |H|=p^m $ . Moreover,
 
   \par
1) Sylow  $ p $ -subgroups exist
   \par
2) Any two Sylow  $ p $ -subgroups are conjugate 
   \par
3) The number  $ n_p $  of Sylow  $ p $ -subgroups satisfies  $ n_p \equiv  1 ( \mathrm{mod}  p) $ .
 
From  (2), we may deduce that
 \[ n_p = | \{ gPg ^{-1} \mid g \in   G \} |=|G(P)|=|G:N_G(P)|  \quad  \text   {which divides } s, \]
where we consider  $ G $  as acting on itself by conjugation.
Thus the most useful information from Sylow's theorem is that
 \[ n_p|s \quad \text{and} \quad n_p \equiv 1 ( \mathrm{mod} p). \]


