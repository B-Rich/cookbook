A non-empty subset  $ I $  of a ring  $ R $  that satisfies
   \par
  $ x,y \in I \implies x-y \in I $  (i.e.   $ (I,+) $  is a subgroup of  $ (R,+)  ;$ 
   \par
  $ x \in I, r \in R \implies rx \in I  . $ 
\par
Equivalently, there is a ring homomorphism from $ R $ which has $ I $
 as its kernel.  Note that this is only itself a ring if it contains $1$ , in which case it is in fact equal to $ R $ by the second
 property.  A nonstandard notation for ''$ I $ is an ideal in $ R $''
 is $ I \triangleleft R $ .
An ideal  $ M $  is  maximal in  $ R $  if
   item  $ M \neq R $  (i.e. $ 1 \notin M $ )
   item  $ M \subseteq I \triangleleft R  \implies I=M  \text{ or } I=R $ .
Equivalently,  $ M $  is maximal iff  $ R/M $  is a field.
An ideal  $ P $  is  prime in  $ R $  if
   \par
  $ 1 \notin P  ; $ 
   \par
  $ ab \in P \implies a \in P  \text{ or } b \in P . $
\par
Note that  $ a \in R $  is prime (sense 1) iff  $ a \neq 0 $  and  $ (a) $ 
is a prime ideal.
See also see principal ideal domain.