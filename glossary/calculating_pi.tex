  There are lots of ways to calculate $ \pi $; for instance
\par
 $ \frac{\pi}{4} = 1 - (1/3) + (1/5) - (1/7) + ... $ (Leibnitz's series)
\par
 $ \frac{\pi}{2} = \frac{2}{1} \cdot \frac{2}{3} \cdot
\frac{4}{3} \cdot \frac{4}{5} \cdot \frac{6}{5} \cdot \frac{6}{7} \cdot
\frac{8}{7} \cdot \frac{8}{9} ... $ (Walliss formula)