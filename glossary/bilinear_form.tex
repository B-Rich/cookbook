Let  $ V $  and  $ W $  be vector
spaces over a field  $ \mathbf{F}. $ A bilinear form on  $ V \times W $  is a function
 $  \varphi : V \times W \to \mathbf{F} $  which is linear in each variable
separately, i.e.  restricting to a single point in either vector
space we obtain a linear map in the other variable. 
\par
For example, if  $ V=W= \mathbf{F}^n $ 
then the map  $  \varphi (v,w)= \sum_{i=1}^n v_iw_i $  is a bilinear form.
\par
Given bases  $ v_1, \ldots,v_n $  for  $ V $  and
 $ w_1, \ldots,w_n $  for  $ W $ , we associate a matrix  $ A $  with bilinear form
 $  \varphi $  by setting  $ A_{ij}= \varphi (v_i,w_j) $ . This means that
 $  \varphi(x,y) $  may be calculated as
 \[ \varphi (x,y)= \sum_{i=1}^n \sum_{j=1}^m x_i y_j A_{ij}=x ^{T} Ab. \]
We can always find bases for  $ V $  and  $ W $  such that the matrix of
 $  \varphi $  is of the form
 \[  \left( 
\begin{array}{cc}
 I_r & 0 \\
 0 & 0
\end{array}   
\right), 
\] where
 $ I_r $  is the  $ r \times r $  identity matrix.