Every square matrix satisfies its own characteristic equation.
\par
ie: if A is a square matrix 
\par
and $ \left | \mathrm{xI - A} \right | = \mathrm{f(x)}, $
\par
(so f(x) is the characteristic
polynomial of A, and f(x)=0 is the characteristic equation of A), 
\par
then the 
Cayley-Hamilton theorem says that f(A) = 0.
\par
eg: Let $ \mathrm{A} = 
\left( \begin{array}{cc} 
  1 & 3 \\ 
  4 & 5
\end{array}
\right) $, then
\par
$ \mathrm{f(x)} = \left | \mathrm{xI - A} \right | $ 
\[ = 
\left| 
\begin{array}{ccc}
( \mathrm{x} -1) & 3 \ 4 & ( \mathrm{x} -5) 
\end{array}
\right| 
\]
\[ = \mathrm{(x-1)(x-5) - 12} \]
\[ = \mathrm{x}^2 -6 \mathrm{x} -7 \]
\par
Then, working in matrices rather than ordinary numbers (so we have 7.I instead of 7),
 $ \mathrm{f(A)} 
= \left( 
\begin{array}{cc}
  1 & 3 \\
  4 & 5
\end{array} 
\right) ^2 - 6 
\left( 
\begin{array}{cc}
  1 & 3 \\
  4 & 5 
\end{array} 
\right) - 
\left(
\begin{array}{cc}
  7 & 0 \\
  0 & 7 
\end{array} 
\right) $
\[ = \left( 
\begin{array}{cc}
  13 & 18 \\
  24 & 37 
\end{array} 
\right) 
- \left( 
\begin{array}{cc} 
  6 & 18 \\
  24 & 30 
\end{array} 
\right) 
- \left( 
\begin{array}{cc}
  7 & 0 \\
  0 & 7 
\end{array} 
\right) \]
\[ = \left( 
\begin{array}{cc}
  0 & 0 \\
  0 & 0 
\end{array}
\right )
\]