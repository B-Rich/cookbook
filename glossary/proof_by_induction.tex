A type of proof: 
\par
We have a hypothesis H, which contains the variable n,
which is a whole number. We want to prove that H is 
true for every value of n. 
\par
1) So we first prove that H is true for n=1.
\par
2) Then we prove that H being true for n=k implies that
H is true for n=k+1.
\par
Then we are done, because 1) and 2) together imply that
H is true for n=2; then 2) implies that H is true for 
n=3; then 2) implies that H is true for n=4... etc. \par
H is called the inductive hypothesis.
\par
Some philosophers do not accept this kind of proof, because
it takes infinitely many steps to prove something. But most 
mathematicians are happy with it.