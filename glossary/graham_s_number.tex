If we use the following notation:
\par $3 \wedge 3= 3^{3}=27$ 
\par $3 \wedge\wedge 3=3^(3^3)=3^27=7,625,597,484,987$
\par $3 \wedge\wedge\wedge 3=3 \wedge\wedge(3 \wedge\wedge 3)=3 \wedge\wedge 7,625,597,484,987 = 3^(7,625,597,484,987^7,625,597,484,987)$
\par $3 \wedge\wedge\wedge\wedge 3 = 3 \wedge \wedge \wedge (3 \wedge \wedge \wedge 3)...$
\par
then we can construct Graham's number as follows:
\par
consider the number $3 \wedge \wedge ...\wedge \wedge \wedge 3$ in which there are $3 \wedge \wedge \wedge \wedge \wedge 3$ arrows;
\par
consider the number in which the number of arrows is equal to the previous 
$3 \wedge \wedge ... \wedge \wedge 3$ number;
\par
continue this process for sixty-three steps.
\par
This number is Graham's number; thought to be the largest ever occurring in
a real mathematical problem (it is listed in the Guinness book of records as
the world's largest number). It occurs in a problem in combinatorics.