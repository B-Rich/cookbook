A square or rectangular set of numbers, usually written enclosed in a large pair of brackets. There are special rules for adding and multiplying matrices that make them handy in representing linear transformations.
\par
For example, the equations 
\begin{eqnarray*}
x + 2y + 3z &=& 1\\
2x + 5y + 7z &=& 2\\
x + 4y &=& 3\\
\end{eqnarray*}
can be represented by the single matrix equation
$$
\mathbf{Ax} = \mathbf{B}
$$
where
$$ \mathbf{A} = \left( 
\begin{array}{ccc}
  1 & 2 & 3 \\
  2 & 5 & 7 \\
  1 & 4 & 0 
\end{array} 
\right )\mbox{,}\quad 
\mathbf{x} = \left(
\begin{array}{c}
  x \\
  y \\
  z 
\end{array} 
\right )\mbox{,}\quad\mbox{and~}   
\mathbf{B} = \left(
\begin{array}{c}
  1 \\
  2 \\
  3 
\end{array} 
\right ).
$$
Matrices can represent many physical quantities which have multiple components in a simple but useful form. For example, the various moments of inertia that a solid has about different axes of rotation can be expressed in a single $3 \times 3$ matrix.
  