Also called the mid-ordinate rule.
\par
If the area under a curve y=f(x) is divided into vertical strips,
then the area of the strip between x=a and x=b is approximately 
equal to $ \mathrm{(b-a)f} \left ( \frac{\mathrm{a+b}}{2} \right ), $ i.e. the 
width of the strip multiplied by its length at its midpoint.
\par
This can be extended into a rule for approximate numerical integration:
\[ \int _a ^b y dx \approx h ( y_{\frac{1}{2}} + y_{1 \frac{1}{2}} 
+ \ldots + y_{n- \frac{1}{2}} ) , \]
where $x_{i}$  divide [a,b] into n pieces of width h=(b-a)/n, 
with $x_{0}=a and x_{n}=b, and x_{i+1/2}$  are
the midpoints of these pieces, and $y_{k}=y(x_{k}).$ 
