To multiply two matrices together, we proceed as follows.
\par
To find the value in row 1, column 1 of the answer, consider 
row 1 of the first matrix and column 1 of the second matrix. Multiply
the first entry in the row by the first entry in the column, the second entry in
the row by the second entry in the column, etc. Then add together the results 
of these multiplications. This gives the number in row 1, column 1 of the answer.
\par
To find the value in row 2, column 1 of the answer, consider
row 2 of the first matrix and column 1 of the second matrix. Multiply
the first entry in the row by the first entry in the column, the second entry in
the row by the second entry in the column, etc. Then add together the results 
of these multiplications. This gives the number in row 2, column 1 of the answer.
\par
Continue in this way to find all of the elements in the answer.
\par
For example: \[ 
\left ( 
\begin{array}{cc} 
  1 & 2 \\
  3 & 4 
\end{array} 
\right ) 
\left ( 
\begin{array}{cc} 
  5 & 6 \\
  7 & 8 
\end{array} 
\right )  =
\left ( 
\begin{array}{cc} 
  (1 \times 5) + ( 2 \times 7 ) & ( 1 \times 6 ) + ( 2 \times 8 ) \\
  ( 3 \times 5) + ( 4 \times 7 ) & ( 3 \times 6 ) + ( 4 \times 8 ) 
\end{array}
\right ) \]
\[ = \left( 
\begin{array}{cc}
  5 + 14 & 6 + 16 \\
  15 + 28 & 18 + 32 
\end{array} 
\right ) \]
\[ = \left ( 
\begin{array}{cc}
  19 & 22 \\
  43 & 50 
\end{array} 
\right ) 
\]