We know that $(x+y)(x-y)=x^{2}-xy+xy-y^{2}$  (by multiplying out the brackets).
\par
But this is the same as $x^{2}-y^{2}.$  So whenever we have an
expression like $x^{2}-y^{2}$  we can write it as (x+y)(x-y). This result
is often useful for simplifying algebra; it is called the difference of
two squares.
\par
For example: $6^2-5^2=(6+5)(6-5)=11 \times 1 =11.$ 
\par 
$17^2-15^2=(17+15)(17-15)=32 \times 1 =64.$