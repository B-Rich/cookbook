The equations which govern the behaviour of electrical
and magnetic fields:
\begin{eqnarray*}
  \nabla . \mathrm{B} &=& 0 \\
  \nabla \times \mathrm{E} + \frac{\partial \mathrm{B}}{\partial \mathrm{t}} &=& 0 \\
  \nabla . \mathrm{D} &=& \rho \\
  \nabla \times \mathrm{H} - \frac{\partial \mathrm{D}}{\partial \mathrm{t}} &=& 
\mathrm{J} \
\end{eqnarray*}
(Factors of $c$, etc, may arise in the equations depending on the system of
units. The above are for MKSA units.)
\par
$\mathrm{E}$ is the electric field.
\par
$ \mathrm{D} = \epsilon_{0} \mathrm{E} + \mathrm{P} $
is the displacement where $ \epsilon_{0} $ is the permittivity of free space and $\mathrm{P}$ is the
polarisation.
\par
$\mathrm{B}$ is the magnetic induction or flux density,
\par
$ \mathrm{H} = \frac{\mathrm{B}}{\mu
_{0}} - \mathrm{M} $ is the magnetic field where $ \mu _{0} $ is the permeability of free space and $\mathrm{M}$ is the magnetisation.
\par
 $ \rho $ is the electric charge density.
\par
 $\mathrm{J}$ is the current density.
\par
In a linear isotropic medium $\mathrm{P}=\mathrm{M}=0$, and $ \epsilon $ and $ \mu $ are constants, 
but in general $\mathrm{H}$ and $\mathrm{D}$ will depend on the earlier behaviour of the system.
\par
Note also that $ \frac{1}{\epsilon_{0} \mu_{0}} = c^2 $ where $c$ is
the speed of light in a vacuum.