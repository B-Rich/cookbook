The position of an astronomical object can be described by two numbers, corresponding
to latitude and longitude on the earth's surface. One way of doing this is using
right ascension and declination:
\par
We call the circle of the sun's path in the sky, the ecliptic. This corresponds to the
earth's equator (but it is \textbf{not} parallel to it). Then the declination is the 
angle that an object is away from the ecliptic (corresponding to latitude), and
the right ascension is how far round the ecliptic the object is, starting from
a fixed point which is (usually) what is called the first point in Aries.