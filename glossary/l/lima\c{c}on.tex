A lima\c{c}on is formed as follows: take a fixed circle of radius a, 
and roll another circle, with the same radius, around the outside of it. 
Take a point P in the disc moving with the moving circle, a distance k from
the circle's edge. The locus of P as the circle rolls, is a lima\c{c}on.
\par
If k=a, then we get a trisectrix, and if k=2a we get a cardioid.
\par
The same curve can be generated as a conchoid with respect to a circle
and a point on the circle.
\par
It is also the inverse of a conic section with respect to one of its  
foci; and it is the pedal curve of a circle with respect to a point in the plane.
\par
It has polar equation $ \mathrm{r=k-2a.cos} \theta . $
\par
It is named after Etienne Pascal, not his son Blaise.
It is the envelope of circles whose centres lie on a fixed circle
and which pass through a given fixed point.