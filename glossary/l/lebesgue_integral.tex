Given a bounded measurable function f(x) over a set E with finite measure,

the Lebesgue integral can be defined:

\par
Let U and L be the upper and lower bounds of f(x) for x in E;

let $t_{0}...t_{n}$  be a dissection of [L, U], 

$ L=t_0 \lt t_1 \lt \ldots \lt t_n = U . $

Now let $e_{i}$  be the subset of E for which 

$ t_{i-1} \leq f(x) \leq t_i . $

\par
Let the Lebesgue measure of $e_{i} be m(e_{i}).$ 

Then the Lebesgue integral of f(x) over E is defined as either of

$ \sum _{i=1} ^{n} t_{i-1}m(e_i) $ or $ \sum _{i=1} ^n

t_i m(e_i) $ as $ \text{max} ( t_i - t_{i-1} ) \to 0 . $

\par
A function that is Riemann integrable is necessarily

Lebesgue integrable.