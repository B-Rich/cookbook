Given a rectangular hyperbola, choose a point P on it. Draw a tangent to the hyperbola
at P. Draw a line perpendicular to the tangent, through the origin. Call the point
where this line intersects the tangent, Q. The locus of Q is called a lemniscate.
\par
This curve has an equation of the form: $(x^{2}+y^{2})^{2}=a^{2}(x^{2}-y^{2}).$ 
\par
In polar coordinates, it has equation 
$ \mathrm{r}^2 = \mathrm{a}^2 \mathrm{cos}2 \theta. $
\par
It is the envelope of the circles whose centres lie on a given 
rectangular hyperbola and which pass through the hyperbola's centre.
\par
It is the inverse of the hyperbola with respect to its centre.
\par
It is a special
case of Cassini's ovals.