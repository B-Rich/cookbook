In linear maths: A map  $  \tau  : V \to W $  between vector spaces satisfying
   \par
  $  \tau (v+w)= \tau (v)+ \tau (w) $ 
   \par
  $  \tau ( \lambda v)= \lambda  \tau (v) $ 
\par
or equivalently just
 $  \tau ( \lambda v+ \mu w)= \lambda  \tau (v)+ \mu  \tau (w) $ , for all
scalars  $  \lambda, \mu $  and all  $ v,w \in V $ . Note that  $ V $  and  $ W $  must
be defined over the same field  $\mathbf{F}$  for this to make any sense.
If  $ V=W $ ,  $  \tau  $  is called an  endomorphism.
\par
Every linear map between finite-dimensional vector spaces corresponds
to a matrix, given an isomorphism between the vector spaces and
 $  \mathbf{F}^n $ . This is done by setting the  $ j $ th column of the matrix to be
the image of the  $ i $ th basis vector of  $  \mathbf{F}^n \cong V $ .
The  kernel of  $  \tau  $  is the set of all values taken by  $  \tau  $ 
to the zero element of  $ W $ . This is clearly a subspace of  $ V $ ,
and similarly the image  $  \tau (V) $  is a subspace of  $ W $ .
The  rank  $  rk \tau   $  of  $  \tau  $  is the dimension of its image
space (i.e.  the set  $  \tau (V) \subseteq W $ ). The  nullity  $  null  \tau  $ 
 is the
dimension of the kernel; the  rank-nullity formula
 \[ rk  \tau + null  \tau = dim V \]
relates these quantities.
See also n-linear, isomorphism, Jordan normal form,
minimum polynomial, characteristic polynomial,
nilpotent