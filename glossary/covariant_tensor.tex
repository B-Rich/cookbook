Given two co-ordinate systems $ \mathrm{x}^{\mathrm{i}} $ and 
$ \bar{\mathrm{x}}^{\mathrm{i}} , $
a covariant tensor of order 1 is a set of components denoted $A_{i}$  
(with respect to $ \mathrm{x}^{\mathrm{i}} ) $ or $ \bar{\mathrm{A}}_{\mathrm{i}} $
(with respect to $ \bar{\mathrm{x}}^{\mathrm{i}} )  , $
and satisfying:
\[ \bar{\mathrm{A}}_{\mathrm{i}} 
= \sum _{\mathrm{r}} \frac{\partial \mathrm{x}^{\mathrm{r}}}
{ \partial \bar{\mathrm{x}}^{\mathrm{i}}}
\mathrm{A}_{\mathrm{r}} , \]
for every i.
\par
A covariant tensor of order 2 is a set of components denoted $A_{ij}$ 
or $ \bar{\mathrm{A}}_{\mathrm{ij}} , $ and satisfying:
\[ \bar{\mathrm{A}}_{\mathrm{ij}} =
\sum _{\mathrm{r}} \sum _{\mathrm{s}}
\frac{\partial \mathrm{x}^{\mathrm{r}}}
{\partial \bar{\mathrm{x}}^{\mathrm{i}}}
\frac{\partial \mathrm{x}^{\mathrm{s}}}
{\partial \bar{\mathrm{x}}^{\mathrm{j}}}
\mathrm{A}_{\mathrm{rs}} , \]
for ever i and j.
\par
Higher-order covariant tensors can be defined similarly.