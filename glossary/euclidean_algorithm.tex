A way of finding the highest common factor of 
two different numbers:
\par
1) Divide the larger number by the smaller. 
\par
2) If the division can be done exactly then the 
HCF is the smaller number. If there is a remainder
then the HCF is the same as the HCF of the remainder
and the smaller number - so repeat step 1 with 
these two numbers.
\par
For instance, to find HCF(17, 20):
\par
20/17 = 1, with remainder 3
\par
so the HCF is the same as HCF(17, 3):
\par
17/3 = 6 with remainder 1
\par
so the HCF is the same as HCF(3, 1):
\par
3/1 = 3; this is exact, so the HCF is 1.