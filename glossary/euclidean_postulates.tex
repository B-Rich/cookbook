Euclid's five basic assumptions, on which he built his geometry:
\par
A straight line can be drawn between any pair of points;
\par
A straight line may be continued indefinitely in any direction;
\par
For any centre, and any radius, a circle may be drawn;
\par
All right angles are equal;
\par
Given a line and a point not on the line, exactly one line may 
be drawn through the given point parallel to the given line.
\par
The last of these, also known as Playfair's axiom, is independent
of the others and may be altered to give different, non-Euclidean geometries.
It was originally stated in a different form, equivalent to:
\par
Given straight lines A and B, and a line C intersecting them, if
the interior angles formed on the same side of C total less than two 
right angles, then A and B produced must meet on this side of C.