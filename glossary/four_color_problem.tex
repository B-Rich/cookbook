The problem which asks whether there exists any network of lines and points
in a plane which requires more than four colours in order to colour it in
such a way that adjacent regions always have different colours.
\par
In 1976 it was proved that four colours is always sufficient to do this, 
but the proof involved using a computer to check a large number of possible 
small networks - it would have taken too long for a human to check the
proof, so some mathematicians feel that this proof is not acceptable.
\par
On different kinds of surfaces the number of colours required is different.
On the surface of a torus it is seven.