A test which is used to measure how well a given set of observations fit a
particular discrete distribution. If the theoretical distribution says that
value $x_{i} should occur E_{i}$  times (expected number of times), and it actually occurs $O_{i}$  times (observed number of times), then:
\[ \chi ^2 = \sum _{\mathrm{i}} \frac{ ( \mathrm{O}_{\mathrm{i}}
- \mathrm{E}_{\mathrm{i}} ) ^2 } { \mathrm{E}_{\mathrm{i}}} \]
has a chi-squared distribution with n-p degrees of freedom, where
p is the number of distribution parameters used in calculating the $E_{i}.$ 
\par
So if $ \chi ^2 $ is large, we reject the hypothetical distribution - e.g.
if it is large enough to have probability smaller than 5 percent, we would
say that the test gave a result that was significant at the 5 percent level, and we
would probably reject the hypothesis.
\par
The test can be used to test for a continuous distribution, by grouping 
the data - but we must be careful not to obtain spurious results by grouping the
data arbitrarily.
\par
Also if several of the $E_{i}$  are smaller than about 5, the test
cannot be used without modifications.