If we start with a line of length 1, and remove the middle 1/3; then
we remove the middle 1/3 of each of the pieces that are left, and then remove the
middle 1/3 of each of the pieces that are left... etc, then the set that we are left
with after continuing this process forever is called the Cantor set.
\par
More strictly, the Cantor set can be defined as follows:
\par
Begin with the closed interval [0,1]. Call this $C_{1}.$ 
\par
Now, remove the middle 1/3 of $C_{1},$  i.e. remove 
the open interval (1/3,2/3) from $C_{1}.$  What remains is
called $C_{2}.$  
\par
$C_{2}$  consists of two closed intervals. Remove the middle 1/3 
of each of them, i.e. remove the open intervals (1/9,2/9) and 
(7/9,8/9). What remains is called $C_{3}.$ 
\par
Continue in this way.             
\par
The Cantor set is the result of continuing this process infinitely far.
\par
More strictly, the Cantor set is the intersection of all $C_{i}$ 
from i=0 to infinity. (But note that $C_{i+1} is contained in C_{i},$ 
so the Cantor set can be thought of as $C_{infinity}.$ 
\par
The Cantor set contains uncountably many points.