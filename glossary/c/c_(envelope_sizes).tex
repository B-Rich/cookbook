A certain set of envelope sizes in common use are called the C-series and designated C4, C3, etc. They are defined as follows:
\par
Each envelope size is a rectangle with its sides in the ratio $ 1 : \sqrt{2} . $ C0 has area $ \sqrt[4]{2} \mathrm{m}^2 ,$ C1 half of this, C2 a quarter, etc.
\par
So the sizes are:
\par
C0 = 917 \ensuremath{ \times } 1297 mm;
\par
C1 = 648 \ensuremath{ \times } 917 mm; 
\par
C2 = 458 \ensuremath{ \times } 648 mm;
\par
C3 = 324 \ensuremath{ \times } 458 mm;
\par
C4 = 229 \ensuremath{ \times } 324 mm;
\par
C5 = 162 \ensuremath{ \times } 229 mm;
\par
C6 = 114 \ensuremath{ \times } 162 mm;
\par
C7 = 81 \ensuremath{ \times } 114 mm;
\par
C8 = 57 \ensuremath{ \times } 81 mm;
\par
C9 = 40 \ensuremath{ \times } 57 mm;
\par
C10 = 28 \ensuremath{ \times } 40 mm.
\par
Size C\emph{n} has width $2^{-1/8-n/2}$ m and height $2^{3/8-n/2}$ m.
\par
The sizes are rounded down to ensure that two C4 envelopes are slightly smaller than one A3 sheet, etc. 
\par
The height and width of size C\emph{n} are the geometric means of the heights and widths of paper sizes A\emph{n} and B\emph{n}, so that A4 fits comfortably into C4, etc.
  