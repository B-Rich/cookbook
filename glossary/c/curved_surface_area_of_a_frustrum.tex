If a frustrum is formed by taking a cone of height h and base radius r,
and slicing the point off to make a frustrum of height a, then:
\par
The curved surface area of the original cone was $ \pi \mathrm{r}
\sqrt{ \mathrm{r}^2 + \mathrm{h}^2 } . $ The curved surface area
of the piece chopped off (another cone) was $ \pi \mathrm{q} \sqrt{ \mathrm{q}^2
+ ( \mathrm{h-a} )^2 } , $ where q is the base radius of the chopped-off cone, i.e. the
radius of the smaller end of the frustrum: q=r(h-a)/h.
\par
So the curved surface area of the frustrum is the difference between these two areas:
$ \mathrm{A} = \pi [ \mathrm{r} \sqrt{\mathrm{r}^2 + \mathrm{h}^2} 
- \mathrm{q} \sqrt{\mathrm{q}^2 + ( \mathrm{h-a} ) ^2 } ] . $
\par
If we only know the height of the frustrum, and the radii of its two ends, a, r and q,
we can find h: h=ar/(r-q) and use the above formulae.