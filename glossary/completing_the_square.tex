This is the method for solving any quadratic equation.

\par
Say we have $ \mathrm{ax}^2 + \mathrm{bx + c} = 0. $
\par
Then divide by a to get: $ \mathrm{x}^2 + \frac{\mathrm{bx}}{\mathrm{a}} 
+ \frac{\mathrm{c}}{\mathrm{a}} = 0. $
\par
Then notice that $ ( \mathrm{x} + \frac{\mathrm{b}}{2 \mathrm{a}} )^2 
= \mathrm{x}^2 + \frac{\mathrm{bx}}{\mathrm{a}} + \frac{\mathrm{b}^2}{4 \mathrm{a}^2} $
\par
so: $ ( \mathrm{x} + \frac{\mathrm{b}}{2 \mathrm{a}} )^2 
- \frac{\mathrm{b}^2}{4 \mathrm{a}^2} + \frac{\mathrm{c}}{\mathrm{a}} = 0, $
\par
 $ ( \mathrm{x} + \frac{\mathrm{b}}{2 \mathrm{a}} )^2 
= \frac{\mathrm{b}^2}{4 \mathrm{a}^2} 
- \frac{\mathrm{c}}{\mathrm{a}}. $
\par
So, $ \mathrm{x} = \frac{ - \mathrm{b} \pm \sqrt { \mathrm{b}^2 - 4 \mathrm{ac}} }
{2 \mathrm{a}}. $
\par
For example, suppose we have $3x^{2}+2x-1=0.$ 
\par
Then we divide by 3 to get: $x^{2}+(2/3)x-(1/3)=0.$ 
\par
Then notice that $(x+(1/3))^{2}=x^{2}+(2/3)x+(1/9)$ 
\par
so: $(x+(1/3))^{2}-(1/9)-(1/3)=0,$ 
\par $(x+(1/3))^{2}=(1/9)+(1/3)=4/9.$ 
\par
So, $ \mathrm{x}+(1/3)= \pm 2/3 $
\par
 $ \mathrm{x} = -1/3 \pm 2/3 $
\par
 x=-1 or x=1/3.