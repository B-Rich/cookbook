This is a type of proof where we start off by assuming
that what we want to prove is untrue. Then we work from
this assumption, and deduce whatever things we can. Hopefully,
sooner or later we will come to something that is obviously
untrue or contradictory. Then we will know that our
original assumption must have been wrong, because it has
led to a contradiction - so we know that the thing we are
trying to prove is actually true.
\par
For example: to prove that there is no such thing as the
highest whole number.
\par
Assume that there is such a number, and call it N.
\par
Then take N, and add 1; call this M. Then M is a whole
number, and M>N, which contradicts our assumption that
N is the largest whole number. So we know that it
must have been wrong to say that there is a biggest whole
number - which is what we meant to prove.